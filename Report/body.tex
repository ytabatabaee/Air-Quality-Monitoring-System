\section{شرح پروژه}
در این پروژه ما یک سامانه‌ی 
\lr{iot}
رصد وضعیت هوا طراحی کردیم که وضعیت 3 نوع گاز سمی در هوا را اندازه‌‌گیری می‌کند و با توجه به مقادیر این گاز‌ها اخطار‌های لازم را به کاربران می‌دهد. این سامانه اطلاعات وضعیت هوا را از تعدادی دستگاه که هر یک سه سنسور مخصوص به گاز ‌دارند می‌گیرد و میانگین وضعیت آن‌ها را نشان می‌دهد و بر اساس وضعیت گاز با بیشترین غلظت وضعیت آلودگی کلی را نشان می‌دهد. این سامانه به صورت یک اپلیکیشن گوشی پیاده‌سازی شده است که با یک لوکیشن خاص کار می‌کند و همچنین یک وب‌سرور که اطلاعات میانگین چند دستگاه را نمایش می‌دهد. شمای کلی طراحی را در شکل زیر مشاهده می‌کنید. 
\begin{figure}[h!]
	\centering		
 	\includegraphics[width=0.8\linewidth]{figs/design.png}
	\caption{شمای کلی سیستم}
\end{figure}


\subsection{تغییرات نسبت به پروپوزال اولیه}
در پروپوزال پروژه ما سیستم را به صورت یک دستگاه در نظر گرفتیم اما در ارائه‌ی اولیه تصمیم بر آن شد که قابلیت داشتن دستگاه در چند لوکیشن مختلف را نیز پیاده‌سازی کنیم. بنابراین ما یک وب سرور طراحی کردیم که بتواند مقادیر گاز‌ها را از چند برد مختلف در لوکیشن‌های مختلف بگیرد و میانگین آن‌ها را گزارش دهد که مشابه کاری است که دستگاه‌های رصد هوای مزاکر مناطق مختلف شهر انجام می‌دهند. همچنین اپلیکیشن گوشی نیز طراحی کردیم که مطابق آنچه در پروپوزال گفته شده بود از یک دستگاه آمار را می‌گیرد و مشابه یک سامانه رصد هوای خانگی برای ساختمان‌های هوشمند عمل می‌کند.  

\section{طراحی و پیاده سازی}

در این پروژه ما با استفاده از 
\href{https://labcenter.s3.amazonaws.com/downloads/iotHelp.pdf}{iot-builder}
مدار‌ها را طراحی کردیم. این ابزار برای سیمولیشن برد‌هایی که برای کاربرد‌های 
\lr{iot}
استفاده می‌شوند و نیاز به استفاده از 
\lr{wifi}
 دارند به کار می‌رود و قابلیت‌های گوناگونی دارد.  در ادامه بخش‌های مختلف طراحی سیستم را شرح می‌دهیم. 
\subsection{شماتیک مدار}
شماتیک مدار را در شکل زیر مشاهده می‌کنید. در این مدار یک برد 
\lr{Arduino Yun}
که برای کاربرد‌های 
\lr{iot}
مناسب است و سه سنسور
\lr{LM35DZ}
تعبیه شده است که سنسور‌های دما هستند اما ما از آن‌ها به عنوان سنسور تشخیص گاز استفاده می‌کنیم. در پروپوزال ما سنسور‌های واقعی اندازه‌گیری گاز 
\lr{MQ-7}
و 
\lr{MQ-2}
و
\lr{MQ-135}
را استفاده کرده بودیم اما چون این سنسور‌ها در 
\lr{Proteus}
نبودند ما از سنسور‌های جایگزین اندازه‌گیری دما استفاده کردیم.
\begin{figure}[h!]
	\centering		
	\includegraphics[width=\linewidth]{figs/circuit.png}
	\caption{شماتیک مدار}
\end{figure}

\subsection{توضیح کلی کد}
کد اصلی هر دستگاه به زبان 
\lr{arduino}
زده شده است و در فایل 
\lr{main.ino}
قرار دارد. شیوه‌ی کلی کار کد هم این است که ابتدا اعداد سنسور‌ها از پورت‌هایشان خوانده می‌شود و سپس با تبدیل زیر به مقدار مناسب در بازه‌ی سنسور گاز‌های هوا در می‌آیند. 
\begin{latin}
	\begin{lstlisting}[language=bash]
s = (s * 500) / 308;
	\end{lstlisting}
\end{latin}
سپس عدد هر گاز روی 
\lr{speedometer}
معادل آن نمایش داده می‌شود و تغییرات لحظه‌ای گاز‌ها نیز روی نمودار با سه خط رنگی ثبت می‌شود. همچنین ماکزیمم مقدار گاز‌های به عنوان معیار اخطار در نظر گرفته می‌شود و یکی از 6 مربع رنگی که در صفحه موجود است پررنگ می‌شود. 

\subsection{اپلیکیشن گوشی}
برای طراحی اپلیکیشن گوشی ما از قابلیت‌های 
\lr{Visual Designer}
استفاده کردیم و از کنترلر‌های 
\lr{iot}
برای نمودار و اسپیدومتر و دیگر اجزای پنل کاربری استفاده کردیم. سپس آپدیت مقادیر این اجزا را در کد انجام دادیم. قابلیت خوب 
\lr{iotbuilder}
این بود که پس از قرار دادن این کنترلر‌ها در پنل گرفیکی به طور اتوماتیک المنتی معادل آن در کد ساخته می‌شد و بنابراین روند کد زدن را خیلی راحت می‌کرد. در زیر شمای پنل کاربری را مشاهده می‌کنید. 
\begin{figure}[h!]
	\centering		
	\includegraphics[width=0.6\linewidth]{figs/client.png}
	\caption{پنل گرافیکی}
\end{figure}

\subsection{وب‌سرور و میانگین‌گیری از نتایج چند دستگاه}
برای پیاده‌سازی کار با چند لوکیشن، ما وب‌سروری طراحی کردیم که اطلاعات گاز‌ها را از دستگاه‌‌های مختلف می‌گیرد و میانگین آن‌ها را در نمودار نمایش می‌دهد. این وب‌سرور با یک ایندکس به نام 
\lr{iot}
که در 
\lr{elasticsearch}
ساخته شده کار می‌کند و هر یک از دستگاه‌ها مقادیرشان را در ایندکس پوش می‌کنند و وب‌سرور به صورت دوره‌ای این مقادیر را می‌خواند و از روی میانگین آن‌ها نمودار می‌کشد. همچنین مشابه قبل اخطار‌های مربوط به وضعیت هوا را با استفاده از مقدار ماکزیمم یک گاز نمایش می‌دهیم. در زیر شمای بخش میانگین گیری وب‌سرور را مشاهده می‌کنید. 

\begin{figure}[h!]
	\centering		
	\includegraphics[width=0.8\linewidth]{figs/web.png}
	\caption{وب‌سرور}
\end{figure} 

\section{فایل‌ها و شیوه ‌ی اجرای برنامه}
برای اجرای برنامه مراحل زیر را طی می‌کنیم:
\begin{itemize}
	\item 
	\textbf{راه اندازی سرور الستیک}:
	برای راه‌اندازی وب سرور باید  
	\lr{elasticsearch}
	روی سیستم نصب باشد. ورژن‌های مختلف نرم‌افزار در 
	\href{https://www.elastic.co/downloads/elasticsearch}{اینجا}
	موجود است. پس از دانلود و 
	\lr{unzip}
	کردن وارد پوشه می‌شویم فایل 
	\lr{elasticsearch.yml}
	را با فایلی که پوشه‌ی 
	\lr{config}
	از پروژه قرار دارد جایگزین می‌کنیم. این فایل قابلیت 
	\lr{cors}
	را به سرور اضافه می‌کند. 
	
	 سپس دستور 
\begin{latin}
\begin{lstlisting}[language=bash]
elasticsearch-7.8.0> .\bin\elasticsearch
\end{lstlisting}
\end{latin}
را می‌زنیم تا سرور راه بیفتد. در نهایت با دستور زیر فایل 
\lr{index.py}
را ران می‌کنیم.  
\begin{latin}
\begin{lstlisting}[language=bash]
python elasticsearch\index.py
\end{lstlisting}
\end{latin}
تا یک ایندکس خالی به نام 
\lr{iot}
در سرور الستیک ساخته شود (روی یک سیستم فقط یک بار نیاز است این دستور اجرا شود). 
	\item 
	\textbf{فایل‌های \lr{Arduino Yun}}:
	پس از باز کردن پروژه در پروتئوس پوشه‌ای با نامی مشابه زیر در آدرسی مشابه زیر ساخته می‌شود:
\begin{latin}
\begin{lstlisting}[language=bash]
Users\USERNAME\AppData\Local\Temp\VSM Studio\RANDOM_NUMBER
\end{lstlisting}
\end{latin}	
در این پوشه یک پوشه‌ی دیگر به نام 
\lr{ARDUINO YUN}
قرار دارد که محتوای آن باید با محتوای این پوشه که ما در فایل‌های پروژه قرار دادیم جایگزین شود. 

در این پیاده‌سازی هر یک از 
\lr{device}
ها روی یک پورت از 
\lr{localhost}
اجرا می‌شوند. 
		
	\item 
	\textbf{راه‌اندازی کلاینت گوشی همراه}:
	برای راه‌اندازی کلاینت گوشی همراه باید نرم‌افزار 
	\lr{IoT Controller}
	را از
	\href{https://play.google.com/store/apps/details?id=com.labcenter.iotbuilder&hl=en_GB}{اینجا}
	 نصب کنیم. سپس به صورت زیر در بخش 
	 \lr{Discover}
	 آی پی و پورتی که برد آردینو روی آن در حال اجرا است را وارد کنیم تا گوشی به برد وصل شود. 
	 \begin{figure}[h!]
	 	\centering			 	
	 	\subfloat{\includegraphics[width=0.45\textwidth]{figs/pic2.jpg}}
	 	\hfill	 	
	 	\subfloat{\includegraphics[width=0.42\textwidth]{figs/pic1.jpg}}
	 	\caption{شیوه‌ی اتصال گوشی به برد}	
	 \end{figure}
	
	\item
	\textbf{مشاهده نتایج در وب‌سرور}:
	برای مشاهده‌ی نتایج در وب‌سرور فایل 
	\lr{web.html}
	در پوشه‌ی 
	\lr{Web}
	را اجرا می‌کنیم. ابتدا میانگین وضعیت کلی دستگاه‌های متصل به سرور و سپس وضعیت هر لوکیشن (هر یک از دستگاه‌ها) به ترتیب مشاهده می‌شود. 
	\begin{figure}[h!]
		\centering		
		\includegraphics[width=0.8\linewidth]{figs/multiloc.png}
		\caption{لوکیشن‌های مختلف به صورت مجزا}
	\end{figure}
	
\end{itemize}

\section{سیمولیشن و نتایج}
ویدیوی کاملی از تست سیستم با چند لوکیشن در فایل‌های پروژه به نام 
\lr{test-video.mp4}
 آورده شده است. 


\section{چالش‌ها}
مشکل اصلی‌ای که در ابتدای پیاده‌سازی با آن مواجه شدیم این بود که محیط شبیه‌سازی مناسب‌ برای کاربرد‌های 
\lr{iot}
که با برد‌های آردوینو کار کند بسیار کم بود و خود 
\lr{iotbuilder}
هم با وجود امکانات زیادش بسته به سیستم ممکن است مشکلاتی را در اجرا ایجاد کند. پیاده‌سازی با برد واقعی بسیاری از این مشکلات را حل می‌کرد. 
\section{منابع}
 لیستی از منابعی که در این پروژه از آن‌ها استفاده کردیم در زیر آوده شده است.
 \begin{latin}
 	 \begin{itemize}
 		\item
 		https://create.arduino.cc/projecthub/vishnutheerth\_e\_p/iot-based-simple-air-pollution-monitoring-system-ce61ad
 		\item 
 		https://labcenter.s3.amazonaws.com/downloads/iotHelp.pdf
 		\item 
 		https://create.arduino.cc/projecthub/east-west-university/indoor-air-quality-monitoring-system-5b5244
 		\item 
 		https://create.arduino.cc/projecthub/vishnutheerth\_e\_p/iot-based-simple-air-pollution-monitoring-system-ce61ad
 		\item
 		https://github.com/Circuito-io/Red-Balloon-Air-Pollution-Detector
 	\end{itemize}
 \end{latin}
